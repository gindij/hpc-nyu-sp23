\documentclass[11pt]{article}
\usepackage[english]{babel}
\usepackage[utf8]{inputenc}

\input{~/Dropbox/Documents/latex_template/preamble/packages}
\input{~/Dropbox/Documents/latex_template/preamble/math}

\title{High Performance Computing: HW1}
\author{Jack Gindi}

\begin{document}

\maketitle

\section*{Problem 1}
I made my mini-presentation on DistGNN on 1/30/2023.

\section*{Problem 2}
The processor I used was the Apple M1 Max. The timing results are shown in the table below:
\begin{table}[h]
    \centering
    \begin{tabular}{|l|l|l|}
    \hline
    \textbf{Dimension} & \textbf{\texttt{-O0} Time (s)} & \textbf{\texttt{-O3} Time (s)} \\ \hline
    100                & 0.328585              & 0.027163              \\ \hline
    200                & 2.602556              & 0.229702              \\ \hline
    300                & 8.786613              & 0.945254              \\ \hline
    400                & 20.864195             & 2.471498              \\ \hline
    500                & 41.699594             & 5.070169              \\ \hline
    \end{tabular}
    \caption{Timings for different matrix sizes with different compiler optimization flags.}
\end{table}

When I used the \texttt{-O0} flag, I observed approximately 0.3 Gflops/s and 0.6 GB/s. When
I used the \texttt{-O3} flag, I observed on average approximately 3 Gflops/s and 6 GB/s, a 10x increase.

\section*{Problem 3}
As in problem 2, I used the M1 Max architecture.

\subsection*{Using \texttt{-O0}}
For $N = 10$, it took about 200 Gauss-Seidel iterations and 0.00003s. For $N = 1000$, it took
almost 2,000,000 in 18s. For $N = 100,000$, it never finished, but I would estimate on
the order of 2 billion ($200,000 \times 10,000$). It took about 10s for 10,000 iterations, so for 2 billion iterations, it would likely take about 555 hours.

\subsection*{Using \texttt{-O3}}
For $N = 10$, it took about 200 Gauss-Seidel iterations and 0.000008s. For $N = 1000$, it took
almost 2,000,000 in 4.5s. For $N = 100,000$, it never finished, but I would estimate on
the order of 2 billion ($200,000 \times 10,000$). It took about 2.5s for 10,000 iterations, so for 2 billion iterations, it would likely take about 138 hours.

\subsection*{Code listing} 
The code used to generate the above data is given below:
\begin{verbatim}
#include <stdio.h>
#include <math.h>
#include "utils.h"

double res_norm(double* u, int n) {
    double h2 = 1. / ((n + 1) * (n + 1));
    double res = 0;
    double Aui;

    for (int i = 0; i < n; i++) {
        Aui = 2 * u[i];
        if (i - 1 >= 0) Aui -= u[i-1];
        if (i + 1 <  n) Aui -= u[i+1];
        Aui /= h2;
        res += (Aui - 1) * (Aui - 1);
    }

    return sqrt(res);
}

void jacobi_step(double **u, double** u_next, int n) {

    double h2 = 1. / ((n + 1) * (n + 1));
    double s;

    for (int i = 0; i < n; i++) {
        s = 0;
        if (i - 1 >= 0) s += -(*u)[i-1];
        if (i + 1 < n)  s += -(*u)[i+1];
        (*u_next)[i] = (h2 - s) / 2;
    }

    for (int i = 0; i < n; i++) {
        (*u)[i] = (*u_next)[i];
    }
}

void gauss_seidel_step(double** u, double** u_next, int n) {
    
    double h2 = 1. / ((n + 1) * (n + 1));
    double s;

    for (int i = 0; i < n; i++) {
        s = 0;
        if (i - 1 >= 0) s += -(*u)[i-1];
        if (i + 1 <  n) s += -(*u)[i+1];
        (*u)[i] = (h2 - s) / 2;
    }
}

int solve(
    int n, 
    int max_iter,
    double epsilon, 
    void (*step_fn)(double**, double**, int), 
    int verbose
) {

    double* u      = (double*) calloc(n, sizeof(double));
    double* u_next = (double*) calloc(n, sizeof(double));

    int iter;
    double init_res_norm = res_norm(u, n);
    for (iter = 0; iter < max_iter; iter++) {
        step_fn(&u, &u_next, n);
        double r_norm = res_norm(u, n);
        if (verbose)
            printf(
                "iteration %d: %5f (decrease factor = %5f)\n", 
                iter + 1, 
                r_norm, 
                init_res_norm / r_norm
            );
        if (r_norm / init_res_norm < epsilon) 
            break;
        iter += 1;
    }

    free(u);
    free(u_next);

    return iter;
}


int main(int argc, char** argv) {

    const int    N        = read_option<int>("-N", argc, argv);
    const int    MAX_ITER = read_option<int>("-maxiter", argc, argv);
    const int    VERBOSE  = read_option<int>("-v", argc, argv);
    const double EPSILON  = 1e-4;
    
    Timer t2;
    t2.tic();
    int iters_gs = solve(N, MAX_ITER, EPSILON, &gauss_seidel_step, VERBOSE);
    double time_gs = t2.toc();
    printf(
        "Gauss-Seidel (N = %d, max iters = %d): time = %f, iters = %d\n", 
        N, 
        MAX_ITER, 
        time_gs, 
        iters_gs
    );

    return 0;
}
\end{verbatim}

\end{document}

